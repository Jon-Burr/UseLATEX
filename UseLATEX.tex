% -*- latex -*-

\documentclass{article}

\newcommand{\UseLATEXVersion}{1.6.1}
\newcommand{\SANDNumber}{SAND 2008-2743P}

% This wonderful package allows hyphenation in tt fonts and hyphenation of
% words with underscores in them.
\usepackage[htt]{hyphenat}

\usepackage{fancyvrb}
\usepackage{color}
\usepackage{xspace}

\usepackage{hyperref}
\hypersetup{pdftitle={UseLATEX.cmake: LaTeX Document Building Made Easy}}
\hypersetup{pdfauthor={Kenneth Moreland}}

% Simple commands that establish the font for various elements.
\newcommand*{\textfile}[1]{\textsf{#1}}
\newcommand*{\textprog}[1]{\textfile{#1}}
\newcommand*{\textlatexpackage}[1]{\textsf{#1}}
\newcommand*{\textcmake}[1]{\texttt{#1}}
\newcommand*{\textcmakevar}[1]{\textcmake{#1}}
\newcommand*{\textmaketarget}[1]{#1}
\newcommand*{\textvar}[1]{\textit{#1}}
\CustomVerbatimCommand{\textlatex}{Verb}{}

% Simple commands that insert some standard text.
\newcommand*{\UseLATEX}{\textfile{UseLATEX.cmake}\xspace}
\newcommand*{\latex}{\LaTeX\xspace}
\newcommand*{\bibtex}{\textsc{Bib}\TeX\xspace}
\newcommand*{\miktex}{Mik\TeX\xspace}
\newcommand*{\ald}{\textcmake{ADD\_LATEX\_DOCUMENT}\xspace}

% Environments for listing CMake and other types of code.
\definecolor{listingframecolor}{cmyk}{0,0,0,0.25}
\CustomVerbatimEnvironment{CodeListing}{Verbatim}{
  frame=single,
  rulecolor=\color{listingframecolor},
  framesep=6pt}
\newcommand*{\includeCodeListing}[2][]{\VerbatimInput[
  frame=single,
  rulecolor=\color{listingframecolor},
  framesep=4pt,#1]{#2}}

\begin{document}

  \sloppy

  \title{UseLATEX.cmake: \latex Document Building Made Easy}
  \author{Kenneth Moreland}
  \date{Version \UseLATEXVersion}
  \maketitle

  \tableofcontents

  %-----------------------------------------------------------------------------

  \section{Description}
  \label{sec:Description}

  Compiling \latex files into readable documents is actually a very
  involved process. Although CMake comes with \textfile{FindLATEX.cmake},
  it does nothing for you other than find the commands associated with
  \latex. I like using CMake to build my \latex documents, but creating
  targets to do it is actually a pain. Thus, I've compiled a bunch of
  macros that help me create targets in CMake into a file I call
  ``\UseLATEX.'' Here are some of the things \UseLATEX handles:

  \begin{itemize}
  \item Runs \latex multiple times to resolve links. 
  \item Can run \textprog{bibtex}, \textprog{makeindex}, and
    \textprog{makeglossaries} to make bibliographies, indexes, and/or
    glossaries.
  \item Optionally runs configure on your \latex files to replace
    \textcmake{@\textvar{VARIABLE}@} with the equivalent CMake variable.
  \item Automatically finds png, jpeg, eps, and pdf files and converts them
    to formats \textprog{latex} and \textprog{pdflatex} understand.
  \end{itemize}

  %-----------------------------------------------------------------------------

  \section{Download}
  \label{sec:Download}

  \UseLATEX is currently posted to the CMake Wiki at
  \begin{quote}
    \href{http://public.kitware.com/Wiki/CMakeUserUseLATEX}{http://public.kitware.com/Wiki/CMakeUserUseLATEX}.
  \end{quote}

  %-----------------------------------------------------------------------------

  \section{Usage}
  \label{sec:Usage}

  Using \UseLATEX is easy. For a basic \latex file, simply include the file
  in your \textfile{CMakeLists.txt} and use the \ald command to make
  targets to build your document. For an example document in the file
  \textfile{MyDoc.tex}, you could establish a build with the following
  simple \textfile{CMakeLists.txt}.

  \begin{CodeListing}
PROJECT(MyDoc NONE)

INCLUDE(UseLATEX.cmake)
ADD_LATEX_DOCUMENT(MyDoc.tex)
  \end{CodeListing}

  The \ald adds the following targets to create a readable document from
  \textfile{MyDoc.tex}:

  \begin{description}
    \item[\textmaketarget{dvi}] Creates \textfile{MyDoc.dvi}. 
    \item[\textmaketarget{pdf}] Creates \textfile{MyDoc.pdf} using
      \textprog{pdflatex}. Requires the \textcmakevar{PDFLATEX\_COMPILER}
      CMake variable to be set.
    \item[\textmaketarget{ps}] Creates \textfile{MyDoc.ps}. Requires the
      \textcmakevar{DVIPS\_CONVERTER} CMake variable to be set.
    \item[\textmaketarget{safepdf}] Creates \textfile{MyDoc.pdf} from
      \textfile{MyDoc.ps} using \textprog{ps2pdf}. Many publishers prefer
      pdfs are created this way. Requires the
      \textcmakevar{PS2PDF\_CONVERTER} CMake variable to be set.
    \item[\textmaketarget{html}] Creates html pages. Requires the
      \textcmakevar{LATEX2HTML\_CONVERTER} CMake variable to be set.
  \end{description}

  One caveat about using \UseLATEX is that you are required to do an
  out-of-source build. That is, CMake must be run in a directory other than
  the source directory. This is necessary as latex is very picky about file
  locations, and the relative locations of some generated or copied files
  can only be maintained if everything is copied to a separate directory
  structure.

  \subsection{Using a Bibliography}
  \label{sec:UsingABibliography}

  For any technical document, you will probably want to maintain a \bibtex
  database of papers you are referencing in the paper. You can incorporate
  your .bib files by adding them after the \textcmake{BIBFILES} argument to
  the \ald command.

  \begin{CodeListing}
ADD_LATEX_DOCUMENT(MyDoc.tex BIBFILES MyDoc.bib)
  \end{CodeListing}

  This will automatically add targets to build your bib file and link it
  into your document. To use the \bibtex file in your \latex file, just do
  as you normally would with \textlatex|\cite| commands and bibliography
  commands:

  \begin{CodeListing}
\bibliographystyle{plain}
\bibliography{MyDoc}
  \end{CodeListing}

  You can list as many bibliography files as you like.

  \subsection{Incoporating Images}
  \label{sec:IncoporatingImages}

  To be honest, incorporating images into \latex documents can be a real
  pain. This is mostly because the format of the images needs to depend on
  the version of \latex you are running (\textprog{latex}
  vs. \textprog{pdflatex}). With these CMake macros, you only need to
  convert your raster graphics to png or jpeg format and your vector
  graphics to eps or pdf format. Place them all in a common directory
  (e.g. images) and then use the \textcmake{IMAGE\_DIRS} option to the \ald
  macro to point to them. \UseLATEX will take care of the rest.

  \begin{CodeListing}
ADD_LATEX_DOCUMENT(MyDoc.tex BIBFILES MyDoc.bib
                   IMAGE_DIRS images)
  \end{CodeListing}

  If you want to break up your image files in several different
  directories, you can do that, too. Simply provide multiple directories
  after the \textcmake{IMAGE\_DIRS} command.

  \begin{CodeListing}
ADD_LATEX_DOCUMENT(MyDoc.tex BIBFILES MyDoc.bib
                   IMAGE_DIRS icons figures)
  \end{CodeListing}

  Alternatively, you could list all of your image files separatly with the
  \textcmake{IMAGES} option.

  \begin{CodeListing}
SET(MyDocImages
  logo.eps
  icons/next.png
  icons/previous.png
  figures/flowchart.eps
  figures/team.jpeg
  )
ADD_LATEX_DOCUMENT(MyDoc.tex IMAGES ${MyDocImages})
  \end{CodeListing}
  %$

  Both the \textcmake{IMAGE\_DIRS} and \textcmake{IMAGES} can be used
  together. The combined set of image files will be processed. If you wish
  to provide a separate eps file and pdf or png file, that is OK,
  too. \UseLATEX will handle that by copying over the correct file instead
  of converting.


  Once you establish the images directory, CMake will automatically find
  all png and eps files in it and add makefile targets to use ImageMagick's
  convert to convert the file times to those appropriate for the build. If
  you do not have ImageMagick, you can get it for free from
  \href{http://www.imagemagick.org}{http://www.imagemagick.org}. CMake will
  also give you a \textcmakevar{LATEX\_SMALL\_IMAGES} option that, when on,
  will downsample raster images. This can help speed up building and
  viewing documents. It will also make the output image sizes smaller.

  One more note about vector graphics. Encapsulated postscript (eps) files
  have a bounding box that is often lost when converting to pdf types. When
  using eps files, it is best to search for a line starting with
  \textcmake{\%\%BoundingBox:} such as

  \begin{CodeListing}
%%BoundingBox: 58 77 734 536
  \end{CodeListing}

  and then copy these numbers to the \textlatex|bb| option of the \latex
  \textlatex|\includegraphics| command:

  \begin{CodeListing}
\includegraphics[width=\linewidth,bb=58 77 734 536]
  \end{CodeListing}

  \subsection{Create a PDF by Default}
  \label{sec:CreateAPDFByDefault}

  By default, when you use \ald and then run make with no arguments, the
  dvi file will be created. You have to specifically build the pdf target
  to use \textfile{pdflatex} to create a pdf file. However, oftentimes we
  want the pdf to be generated by default. To do that, simply use the
  \textcmake{DEFAULT\_PDF} option to \ald:

  \begin{CodeListing}
ADD_LATEX_DOCUMENT(MyDoc.tex BIBFILES MyDoc.bib
                   IMAGE_DIRS images
		   DEFAULT_PDF)
  \end{CodeListing}

  \subsection{Building Multiple \latex Documents}
  \label{sec:BuldingMultipleLatexDocuments}

  The most commen use for \UseLATEX is to build a single document, such as
  a paper you are working on. However, some use cases involve building
  several documents at one time. To do this, you must call \ald multiple
  times. However, if you do this, the dvi, pdf, etc. targets will be
  generated multiple times, and that is illegal in the current version of
  CMake.\footnote{CMake version 2.4 as of this writing.} To get around
  this, you need to mangle the names of the targets that \ald creates. To
  do this, use the \textcmake{MANGLE\_TARGET\_NAMES} option.

  \begin{CodeListing}
ADD_LATEX_DOCUMENT(MyDoc1.tex MANGLE_TARGET_NAMES)
ADD_LATEX_DOCUMENT(MyDoc2.tex MANGLE_TARGET_NAMES)
  \end{CodeListing}

  In the example above, the first call to \ald will create targets named
  \textmaketarget{MyDoc1\_dvi}, \textmaketarget{MyDoc1\_pdf},
  \textmaketarget{MyDoc1\_ps}, etc. whereas the second call will create
  targets named \textmaketarget{MyDoc2\_*}.

  If you still want the simple, short targets to build all of the
  documents, you can add them yourself with custom targets that depend on
  the targets created by \ald

  \begin{CodeListing}
ADD_CUSTOM_TARGET(dvi)
ADD_DEPENDENCIES(MyDoc1_dvi MyDoc2_dvi)
ADD_CUSTOM_TARGET(pdf)
ADD_DEPENDENCIES(MyDoc1_pdf MyDoc2_pdf)
ADD_CUSTOM_TARGET(ps)
ADD_DEPENDENCIES(MyDoc1_ps MyDoc2_ps)
  \end{CodeListing}

  \subsection{Making an Index}
  \label{sec:MakingAnIndex}

  You can make an index in a \latex document by using the
  \textlatexpackage{makeidx} package. However, this package requires you to
  run the \textprog{makeindex} command. Simply add the
  \textcmake{USE\_INDEX} option anywhere in the \ald arguments, and
  \textprog{makeindex} will automatically be added to the build.

  \begin{CodeListing}
ADD_LATEX_DOCUMENT(MyDoc.tex BIBFILES MyDoc.bib
                   IMAGE_DIRS images
		   USE_INDEX)
  \end{CodeListing}

  \subsection{Making a Glossary}
  \label{sec:MakingAGlossary}

  There are multiple ways to make a glossary in a \latex document, but the
  \textlatexpackage{glossaries} package provides one of the most convenient
  ways of doing so.  Like the \textlatexpackage{makeidx} package,
  \textlatexpackage{glossaries} requires running \textprog{makeindex} for
  building auxiliary files.  However, building the glossary files can be
  more complicated as there can be different sets of glossary files with
  different extensions.  \UseLATEX will handle that for you.  Simply add
  the \textcmake{USE\_GLOSSARY} option anywhere in the \ald arguments, and
  the glossary creating will be handled for you.

  \begin{CodeListing}
ADD_LATEX_DOCUMENT(MyDoc.tex BIBFILES MyDoc.bib
                   IMAGE_DIRS images
		   USE_GLOSSARY)
  \end{CodeListing}

  \subsection{Multipart \latex Files}
  \label{sec:MultipartLatexFiles}

  Often, it is convenient to split a \latex document into multiple files
  and use the \latex \textlatex|\input| or \textlatex|\include| command to
  put them back together. To do this, all the files have to be located
  together. \UseLATEX can take care of that, too. Simply add the
  \textcmake{INPUTS} argument to \ald to copy these files along with the
  target tex file. Build dependencies to these files is also established.

  \begin{CodeListing}
ADD_LATEX_DOCUMENT(MyDoc.tex
  INPUTS Chapter1.tex Chapter2.tex Chapter3.tex Chapter4.tex
  BIBFILES MyDoc.bib
  IMAGE_DIRS images
  USE_INDEX
  )
  \end{CodeListing}

  As far as \UseLATEX is concerned, input files do not necessarily have to
  be tex files.  For example, you might be including the contents of a text
  file into your document with the \textlatex|\VerbatimInput| command of
  the \textlatexpackage{fancyvrb} package.  In fact, you could also add
  graphic files as inputs, but you would not get the extra conversion
  features described in Section~\ref{sec:IncoporatingImages}.

  \subsection{Configuring \latex Files}
  \label{sec:ConfiguringLatexFiles}

  Sometimes it is convenient to control the build options of your tex file
  with CMake variables. You can achieve this by using the
  \textcmake{CONFIGURE} argument to \ald.

  \begin{CodeListing}
ADD_LATEX_DOCUMENT(MyDoc.tex
  INPUTS Chapter1.tex Chapter2.tex Chapter3.tex Chapter4.tex
  CONFIGURE MyDoc.tex
  BIBFILES MyDoc.bib
  IMAGE_DIRS images
  USE_INDEX
  )
  \end{CodeListing}

  In the above example, in addition to copying \textfile{MyDoc.tex} to the
  binary directory, \UseLATEX will configure \textfile{MyDoc.tex}. That is,
  it will find all occurrences of \textcmake{@\textvar{VARIABLE}@} and
  replace that string with the current CMake variable
  \textcmakevar{\textvar{VARIABLE}}.

  With the \textcmake{CONFIGURE} argument you can list the target tex file
  (as shown above) as well as any other tex file listed in the
  \textcmake{INPUTS} argument.

  \begin{CodeListing}
ADD_LATEX_DOCUMENT(MyDoc.tex
  INPUTS Ch1Config.tex Ch1.tex Ch2Config.tex
         Ch2.tex Ch3Config Ch3.tex
  CONFIGURE Ch1Config.tex Ch2Config.tex Ch3Config.tex
  BIBFILES MyDoc.bib
  IMAGE_DIRS images
  USE_INDEX
  )
  \end{CodeListing}

  Be careful when using the \textcmake{CONFIGURE} option. Unfortunately,
  the \textlatex|@| symbol is used by \latex in some places. For example,
  when establishing a tabular environment, an \textlatex|@| is used to
  define the space between columns. If you use it more than once, then
  \UseLATEX will erroneously replace part of the definition of your columns
  for a macro (which is probably an empty string). This can be particularly
  troublesome to debug as \latex will give an error in a place that, in the
  original document, is legal. Hence, it is best to only configure tex
  files that contain very little text of the actual document and instead
  are mostly setup and options.

  %-----------------------------------------------------------------------------

  \section{Frequently Asked Questions}
  \label{sec:FrequentlyAskedQuestions}

  This section includes resolutions to common questions and issues
  concerning use of \UseLATEX and with \latex in general.

  \subsection{How do I process \latex files on Windows?}
  \label{sec:How_do_I_process_latex_files_on_Windows}

  I have successfully used two different ports of LaTeX for windows: the
  \href{http://www.cygwin.com/}{cygwin} port
  (\href{http://www.cygwin.com/}{http://www.cygwin.com/}) and the
  \href{http://www.miktex.org/}{\miktex} port
  (\href{http://www.miktex.org/}{http://www.miktex.org/}).

  If you use the cygwin port of \latex, you must also use the cygwin port
  of CMake, make, and ImageMagick. If you use the \miktex port of \latex,
  you must use the CMake from
  \href{http://www.cmake.org/HTML/Download.html}{http://www.cmake.org/HTML/Download.html},
  the ImageMagick port from
  \href{http://www.imagemagick.org/script/index.php}{http://www.imagemagick.org/script/index.php},
  and a native build tool like MSVC or the GNU make port at
  \href{http://unxutils.sourceforge.net/}{http://unxutils.sourceforge.net/}.
  \emph{Do not use the ``native'' CMake program with any cygwin programs or
  the cygwin CMake program with any non-cygwin programs.} This issue at
  hand is that the cygwin ports create and treat filenames differently then
  other windows programs.\footnote{If you are careful, you can use the
  cygwin version of make with the windows ports of CMake, \latex, and
  ImageMagick.  It is an easy way around the problems described in
  Section~\ref{sec:Why_is_convert_failing_on_Windows}.}

  Also be aware that if you have images in your document, there are
  numerous problems that can occur on Windows with the ImageMagick convert
  program. See Section~\ref{sec:Why_is_convert_failing_on_Windows} for more
  information.

  \subsection{How do I process \latex files on Mac OS X?}
  \label{sec:How_do_I_process_latex_files_on_Mac_OS_X}

  I highly recommend using the port of \latex that is available from the
  \href{http://www.finkproject.org/}{fink project}. The easiest way to get
  files from \href{http://www.finkproject.org/}{fink project} is to use
  \href{http://finkcommander.sourceforge.net/}{FinkCommander}
  (\href{http://finkcommander.sourceforge.net/}{http://finkcommander.sourceforge.net/}).

  The ImageMagick \textprog{convert} program is also available from the
  \href{http://www.finkproject.org/}{fink project}.

  \subsection{Why is convert failing on Windows?}
  \label{sec:Why_is_convert_failing_on_Windows}

  Assuming that you have correctly downloaded and installed an appropriate
  version of ImageMagick (as specified in
  Section~\ref{How_do_I_process_LaTeX_files_on_Windows}), there are several
  other problems that users can run into the created build files attempt to
  run the convert program.

  A common error seen is 

  \begin{CodeListing}
Invalid Parameter - filename
  \end{CodeListing}

  This is probably because CMake has found the wrong \textprog{convert}
  program. Windows is installed with a program named \textprog{convert} in
  \textfile{\%SYSTEMROOT\%$\backslash$system32}. This \textprog{convert}
  program is used to change the filesystem type on a hard drive. Since the
  windows \textfile{convert} is in a system binary directory, it is usually
  found in the path before the installed ImageMagick \textfile{convert}
  program. (Don't get me started about the logic behind this.) Make sure
  that the \textcmakevar{IMAGEMAGICK\_CONVERT} CMake variable is pointing
  to the correct \textprog{convert} program.

  Another common error is that \textprog{convert} not finding a file that
  is clearly there.

  \begin{CodeListing}
convert: unable to open image `filename'
  \end{CodeListing}

  If you notice that the drive letter is stripped off of the filename
  (i.e. \textfile{C:}), then you are probably mixing the Cygwin version of
  \textprog{convert} with the non-cygwin CMake. The cygwin version of
  \textprog{convert} uses the colon (:), as a directory separator for
  inputs. Thus, it assumes the output file name is really two input files
  separated by the colon. Switch to the non-cygwin port of ImageMagick to
  fix this.

  If you are using nmake, you may also see the following error: 

  \begin{CodeListing}
convert.exe: unable to open image `C:': Permission denied.
  \end{CodeListing}

  I don't know what causes this error, but it appears to have something to
  do with some strange behavior of nmake when quoting the convert
  executable. The easiest solution is to use a different build program
  (such as make or MSVC's IDE or a unix port of make). If anyone finds away
  around this problem, please contribute back.

  \subsection{Why does make stop after each image conversion?}
  \label{sec:Why_does_make_stop_after_each_image_conversion}

  There is a bug in the ImageMagick convert version 6.1.8 that
  inappropriatly returns a failure condition even when the image convert
  was successful. The problem might also occur in other ImageMagick
  versions. Try updating your installation of ImageMagick.

  %-----------------------------------------------------------------------------

  
  \section{Acknowledgments}

  Thanks to all of the following contributors.

  \begin{description}
  \item[Alin Elena] Suggestions on removing dependence on makeglossaries
    command.
  \item[\O{}ystein S. Haaland] Support for making glossaries.
  \item[Eric Noulard] Support for any extension on \latex input files.
  \end{description}

  This work was primarily done at Sandia National Laboratories.  Sandia is
  a multiprogram laboratory operated by Sandia Corporation, a Lockheed
  Martin Company, for the United States Department of Energy's National
  Nuclear Security Administration under contract DE-AC04-94AL85000.

  This document released as document \SANDNumber.

  %-----------------------------------------------------------------------------

  \appendix

  \section{Sample CMakeLists.txt}
  \label{sec:SampleCMakeLists.txt}

  Following is a sample listing of CMakeLists.txt.  In fact, it is the
  CMakeLists.txt that is used to build this document.

  \includeCodeListing{CMakeLists.txt}

  \section{UseLATEX.cmake Listing}
  \label{sec:UseLATEX.cmakeListing}

  \includeCodeListing[fontsize=\footnotesize]{UseLATEX.cmake}

\end{document}
